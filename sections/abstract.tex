\begin{abstract}
Software development is a complex task that virtually always involves
developers collaborating. In this sense, many Free Libre and Open Source
Software (FLOSS) development models have successfully created large, scalable,
and (more often than not) globally distributed communities that evolve and
maintain high-quality software projects, with the Linux kernel ecosystem being
a prime example. Beyond the technical challenges natural to the project, the
skills and knowledge requirements are high and discouraging for newcomers to
the Linux ecosystem as there are many sub-projects (called subsystems), each
with its specific contributing rules, processes, and practices that are usually
undocumented and only learned by direct contact with the community; this risks
the project long-term sustainability, as the renewal of the highly qualified
workforce is a known problem. This work aims to validate an approach to mentor
newcomers to the Linux ecosystem efficiently preparing them to become real
Linux developers. During an offering of a Free Software Development course
ministered by the authors, students went from setting up a testing environment
to learning the fundamental workflows involved, culminating in sending patches
and interacting with Linux communities through the code review process. These
students, who in the majority had no experience in Linux development, were
closely mentored using a combination of teaching techniques, and their
experience and feedback were collected through surveys and blog posts written
by them. Among our findings, we can highlight: (1) Use of directed content
(tutorials) produced by real practitioners in the Linux ecosystem, along with
in-person mentoring during classes (workshops) and accessibility of the
professor and teaching assistants, produces a fertile environment for
newcomers; (2) The experience in the course enhanced the qualification of
students, from hard skills like git, and email and web-based models of code
collaboration to communication skills; (3) The experience in the course
demystified a lot of inaccurate concepts from the students about FLOSS
development and made them more comfortable and ready to contribute to other
FLOSS projects. We claim that these contributions provide a solid approach on
mentoring newcomers to the Linux ecosystem as well as equipping them with the
necessary skills and experience to become real Linux developers.
\end{abstract}
