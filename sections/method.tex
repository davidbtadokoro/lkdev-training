\section{Method}
\label{sec:method}

The main goal of this work is to design efficient and reproducible guidelines to
train several inexperienced contributors to the Linux kernel ecosystem,
providing them with a solid base of hard and soft skills necessary to become
long-lasting assets in Linux kernel development. The guidelines can be
characterized by answering our two main research questions:

\begin{itemize}
    \item \textbf{RQ1.} What teaching techniques are effective in introducing
    newcomers to the Linux ecosystem (in-loco workshops, tutorials, lectures
    from practitioners)?
    \item \textbf{RQ2.} What hard and soft skills are crucial to success in the
    Linux ecosystem as a contributor?
\end{itemize}

Our research work was divided into two parts:

\begin{enumerate}
    \item Training a group of computer science students which was implemented in
    a four-month university course.
    \item Collection and analysis of qualitative data representing the
    perspectives from students and mentors about the experience of the course.
\end{enumerate}

\subsection{Training Contributors to the Linux ecosystem}

The course was offered at a Computer Science higher education institution and
was attended by undergraduate, postgraduate, and other students. Of all the
students enrolled, 24 completed the course in full.

The course structure focused on practical and interactive activities rather than
lectures or expository classes, even though the latter were seldom used. The
course was divided into the following three phases. Phases one to three occurred
chronologically, and went from the lowest level of the Linux ecosystem, passing
through the supporting and downstream projects, and ending in an arbitrary FLOSS
project.

For most steps of each phase, report-like activities tied to grades were
required to keep track of the student's progress and ensure that the whole class
was roughly on the same page.

\subsubsection{\textbf{Phase 1:} Linux kernel contribution}

In the first phase, students were exposed to the development of the kernel
component of GNU/Linux operating systems. The Linux kernel is the lowest layer
of the Linux ecosystem.

Initially, students spent five weeks building a development setup and learning
the fundamental processes and practices (workflows) of Linux kernel development
using tutorials devised by a veteran Linux practitioner. Each week, there were
two separate in-loco workshop sessions where all students had to do the
tutorials with the support of the mentors. The mentors were present throughout
all sessions to help any student individually, not just by troubleshooting the
several problems encountered over the execution of the tutorials, but also by
answering conceptual doubts and curiosities.

After setting up the necessary environment and knowledge, students spent three
weeks planning and developing a patch destined for a Linux subsystem,
culminating in sending the patch and participating in the review process. A
comprehensive list of patch suggestions was produced by the same veteran Linux
practitioner who authored the tutorials used in part one of the first phase in
the context of the Industrial I/O (IIO) subsystem, but students were not obliged
to follow the suggestions or contribute to IIO. During these three weeks, the
two separate in-loco workshop sessions continued, where mentors helped the
students from understanding a patch suggestion to replying to maintainer
feedback. A guide on how to use git to create, send, and update a patch for
email-based development models (the case of the Linux kernel) made by the
mentors was introduced to the students.

For this phase, complex changes of deep impact were not demanded, as the focus
was to present students with the overall dynamics and skills necessary to
contribute to the Linux kernel, beyond technical knowledge of device drivers or
low-level programming. Nevertheless, naive and straightforward contributions
(like fixing typos or coding style violations) were discouraged as they are
considered spam in most Linux communities and do not bring any challenge from
the perspective of development and reviewing.

\subsubsection{\textbf{Phase 2:} GNU/Linux supporting ecosystem contribution}

In the second phase, students moved from the base layer of the Linux ecosystem
to the tools and projects that support the ecosystem, i.e., projects that
directly help Linux kernel developers in their daily tasks in the form of tools
or similar, and projects that use the Linux kernel as a platform, which in this
context we call \textit{downstream} projects, as they feed of the
\textit{upstream} project (that is the Linux kernel).

Students started this phase collaborating with the
\textit{kworkflow}\footnote{\href{https://kworkflow.org}{https://kworkflow.org}}
project, which is a set of tools that automate and streamline the daily tasks of
Linux developers in a unified interface. Unlike the Linux kernel, kworkflow is a
project hosted on GitHub, so mentors had to introduce students to how Web-based
projects operate and the fundamental concepts to contribute to those. This part
took two weeks, one of which had two workshop sessions (the other week, students
were on a mid-semester break), where mentors helped students in the same fashion
as in the part of sending contributions to the Linux kernel, but adapting it to
the context of the kworkflow project.

After collaborating with kworkflow, students had a workshop conducted by a
veteran Debian developer about Debian packaging, using Perl projects as context.
This part lasted only one week, with two workshop sessions composed of two
hands-on tutorials that introduced students to the overall workflow of Debian
packaging, whilst they attempted to package a Perl project. In this part,
mentors did not have considerable knowledge or experience in software packaging,
so most of the mentoring was done by the guest Debian developer.

Even though contributing to kworkflow and Debian package is arguably simpler and
has less setup overhead, contributions in this phase were still not required to
be of impact.

\subsubsection{\textbf{Phase 3:} Arbitrary FLOSS project contribution}

In the last phase, students chose to contribute to an arbitrary FLOSS project.
Continuing to contribute to the Linux kernel, kworkflow, or Debian packaging was
possible (and even encouraged), but students were not obliged to do so.

In the last 5 weeks, students clustered around a handful of FLOSS projects, with
some repeating projects from the other phases. Workshop sessions continued.
However, the form of those was more unrestrained as there was a significant
variability in the development contexts. In this sense, mentors adopted a more
passive consulting role to help and guide students with any issues relating to
their contributions and projects. At this point, we envisioned that, after
experiencing the considered harder layer of the Linux ecosystem, then a more
understandable layer of the ecosystem, students were more than equipped to
handle entering new FLOSS projects on their own (in case they did not continue
in a project of the previous phases).

Unlike the contributions of the other two phases, students were expected to
contribute more meaningfully to the respective projects.

\subsection{Collecting and analyzing perspectives of students and mentors}

Qualitative data that encodes the perspectives of students and mentors was
collected and analysed to extract the necessary insights for us to answer our
research questions and determine where we got it right and wrong to devise solid
guidelines.

To accomplish this, we used three different sources of informations:

\begin{enumerate}
    \item Students blog posts
    \item Students surveys
    \item Mentors interviews
\end{enumerate}

\subsubsection{Students blog posts}

Throughout the course, students had to present report-like deliverables in the
form of blog posts documenting their experiences in each step of the phases. The
only direction provided to students about how to produce the blog posts was to
aim to do a \textit{board log}. A non-obligatory post summarizing the whole
course experience was asked to be written.

\subsubsection{Students surveys}

We conceived a survey with 47 questions to query for different perspectives of
the students. This survey was released to students at the end of the course, and
they were asked to answer voluntarily.

Questions covered a wide range of topics, from the class characterization
(education level, professional background, familiarity with development tools,
among others) to their feedback on the course activities, materials, and
personnel (were the workshops helpful? was the mentor's assistance useful? which
steps were improductive?) to their individual perspective on their skills
improvement as result of the training (how confident are you in continuing
contributing to the Linux kernel?). Most questions used the Likert scale
(ratings from 1 to 5 in different forms), but some were boolean (yes or no), and
even free form.

\subsubsection{Mentors interviews}

Through direct conversation, the mentor's perspectives were collected. In the
interviews, we did not follow a script, just overall questions about how they
perceived the course experience

\subsubsection{Analyzing the different sources of data}

We surveyed students using the Limesurvey platform and had a full report of the
answers out of the box. We selected meaningful answers and clustered replies to
questions that used the Likert scale as positive, neutral, and negative.

Regarding the mentor's interviews, as they represented a smaller volume of data
than the students' blog posts and surveys, we did not filter or aggregate the
data from this source, taking it at face value.
